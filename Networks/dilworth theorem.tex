Chain: Set of elements in which every two are comparable.

Antichain: Set of elements in which every two are NOT comparable.

The graph is built by making an edge between U and V if U comparable

to V (transitivity applies).

\begin{itemize}
  \item The width of a finite partially ordered set S is the minimum number of chains needed to cover S, i.e. the minimum number of chains such that any element of S is in at least one of the chains.
  \item The width of a finite partially ordered set S is the maximum size of an antichain in S.
  \item The maximum size of an antichain is ( Number of nodes - Maximum Bipartite Matching)
\end{itemize}
